\documentclass{sprz}
\usepackage[backend=bibtex,style=numeric,sorting=none]{biblatex}
\usepackage{enumitem}

\addbibresource{bibliography.bib}

\studfield{Informatyka}
\studtype{Zaoczne}
\title{BatMonit -- system wykrywania nietoperzy na farmach wiatrowych}
\engtitle{BatMonit -- bat detection system on wind farms}
\acronym{Batmonit}
\titledate{2021-11-07}
\supervisor{dr Puźniakowski Tadeusz}
\author{Juliusz Orłowski}{s19799}{Aplikacje Internetowe}{Niestacjonarny}
\author{Jakub Prucnal}{s19800}{Sztuczna Inteligencja}{Niestacjonarny}
\author{Magdalena Wybraniec}{s19798}{Sztuczna Inteligencja}{Niestacjonarny}
\consultant{Dawid Gradolewski}
\consultant{Damian Dziak}
\projectgoals{Projekt typu R\&D, ma na celu wytworzenie systemu softwarowego pobierającego i analizującego dźwięk pochodzący z mikrofonu ultradźwięków oraz rozpoznającego pojawienie się nietoperzy, zapisujący nagrania w bazie danych i wizualizujący je w interfejsie - panelu administratora. Dodatkowo celem będzie przygotowanie analizy odległości i kątów z jakiej mikrofon rejestruje głos nietoperza i adekwatnie – zaproponowanie liczby i rozstawienia urządzeń nasłuchowych, tak aby pokryć kąt 360 stopni wokół turbiny na farmie wiatrowej.}
\productsandservices{System wykrywający nietoperze z nagrań ultradźwięków, możliwy do połączenia z systemem wyłączania turbiny}
\mainfunctionalities{
\begin{itemize}
\item{Mikrofon ultradźwiękowy wraz z oprogramowaniem nagrywającym dźwięk}
\item{Urządzenie do którego nagrania są przesyłane i przechowywane oraz automatycznie analizowane w poszukiwaniu na nim nietoperza}
\item{Baza danych do której trafiają przeanalizowane nagrania}
\item{Interfejs wizualizujący nagrania}
\end{itemize}
}
\successmeasure{Oprogramowanie wykrywające i rozpoznające pojawienie się nietoperzy oraz wytworzenie urządzenia do rejestracji ultradźwięków. Dokonana analiza ulokowania urządzeń rejestrujących na wiatraku oraz jej akceptacja przez konsultanta z firmy BIOSECO.}
\projlimitations{
Czas trwania projektu jest ograniczony do momentu przekazania książki dyplomowej do dziekanatu uczelni PJATK.
Ograniczona dostępność nagrań głosów nietoperzy w full-spectrum – możliwa sytuacja gdy tworzenie projektu będzie odbywało się na podstawie nagrań przetworzonych.
Aktywność nietoperzy ma miejsce od końca marca do października – w związku z tym brak możliwości nagrywania ich aktywności w trakcie semestru zimowego – a tym samym brak możliwości wykonania prób hardware’u przed kwieniem 2022.
}
\date{\today}
\nabstract{TODO}



\begin{document}
\renewcommand{\labelenumii}{\arabic{enumi}.\arabic{enumii}}

\maketitle

\makeprojectcard
\makedeclaration

\tableofcontents

\chapter{Wstęp}\label{ch:wstep}

U podłoża rozwoju alternatywnych źródeł elektryczności leży uczynienie branży energetycznej bardziej „zieloną”, aby życie przyszłych pokoleń było przynajmniej tak samo dobre, jak nasze jest dzisiaj. W poszukiwaniu rozwiązań, które zabezpieczają społeczności w dobie globalnego ocieplenia,  zrównoważony rozwój stał się rdzeniem idei alternatywnych źródeł energii. Druga część historii posiada jednak swą ciemną stronę – wiele danych naukowych wskazuje na znaczący niekorzystny wpływ farm wiatrowych na przyrodę, a zwłaszcza nietoperze. A są one ważnym elementem bioróżnorodności i poszukiwanego zrównoważonego rozwoju. Autorzy pracy dyplomowej w kooperacji z firmą Bioseco wypracowali oparty o sztuczną inteligencję system, który w przypadku dalszego jego rozwijania, zmniejszy śmiertelność nietoperzy na farmach wiatrowych i pomoże utrzymać branży wiatrowej miano ekologicznej. Chiropterologom zaś oraz urzędnikom oceniającym projekty wiatrowe pod kątem realizacji przyrodniczych norm prawnych, zapewni dodatkowe narzędzia minimalizacji wpływu farm na nietoperze.

\section{Cele projektu}

W ramach sformułowanego tematu wyszczególniono następujące cele badawcze i programistyczne:

\begin{itemize}
\item{realizację działającego produktu o minimalnym zestawie funkcjonalności (ang: Minimal Viable Product, dalej: MVP) rejestrującego i identyfikującego nietoperze oraz wysyłającego sygnał wyłączenia – docelowo do systemu turbiny wiatrowej – składającego się z części sprzętowej i oprogramowania: modelu konwolucyjnych sieci neuronowych (ang: convoluted neural network, dalej CNN) oraz z aplikacji użytkownika oraz bazy danych,}
\item{opracowanie koncepcji liczby i układu mikrofonów ultradźwięków docelowego produktu poprzez przeprowadzenie badań terenowych i laboratoryjnych w zakresie zasięgu pracy testowanego sprzętu, tak by cały system pokrywał obszar 360 stopni dookoła turbiny wiatrowej, w odległości do około 100 m,}
\item{opracowanie modelu sieci neuronowych uzyskującego przynajmniej 90\% skuteczności w identyfikacji nietoperzy - borowca wielkiego Nyctalus noctula i karlików Pipistrellus sp., głównych ofiar kolizji z turbinami wiatrowymi,}
\item{przyczynienie się do rozwiązania realnego problemu występującego w obszarze branży energetyki wiatrowej oraz ochrony przyrody,}
\item{rozpoczęcie współpracy z Bioseco Sp. z o.o. w celu zaprezentowania umiejętności studentów firmie oraz ich szybszego rozwijania dzięki kontaktowi z realnymi problemami biznesowymi, merytorycznymi, produkcyjnymi i wdrożeniowymi, rozwiązywanymi aktualnie przed doświadczony zespół specjalistów firmy.}
\end{itemize}

\chapter{Podłoże projektu}

\section{Geneza problemu}

Lądowe farmy wiatrowe mogą mieć znaczący wpływ na środowisko naturalne, w szczególności na nietoperze. Wszystkie gatunki nietoperzy w Polsce są objęte ścisłą ochroną gatunkową i podlegają ochronie prawnej zgodnie z Rozporządzeniem Ministra Środowiska z dnia 16 grudnia 2016 r. w sprawie ochrony gatunkowej zwierząt \cite{Rozporządzenie}. Tym samym inwestor realizujący inwestycję wiatrową przestrzegając prawa ochrony przyrody jest zobligowany uzyskać tzw. Decyzję Środowiskową (dalej: DŚ). Zawarte są w niej wszelkie informacje dotyczące: chiropterofauny danego obszaru, szacowanego wpływu inwestycji na nietoperze, działań zapobiegających i minimalizujących ewentualny wpływ inwestycji na nietoperze, tak by spełniała ona założenia dobrych praktyk i przepisy prawa – polskiego i międzynarodowego. 

Dotychczas przyjętą praktyką w przypadku wykrycia zbyt dużych aktywności nietoperzy w monitoringu przedrealizacyjnym było wpisanie do DŚ obowiązkowych wyłączeń turbin w okresach, w których te zbyt duże aktywności wykryto \cite{Wytyczne}. Jednak dynamika użytkowania przestrzeni przez nietoperze jest bardzo zmienna i po realizacji inwestycji ssaki te mogą mieć bardzo zróżnicowaną aktywność i w całych wyznaczonych okresach nie muszą być zagrożone. A wyłączenia turbin na długie okresy w roku wiążą się z ogromnymi stratami inwestorów. System opracowany w ramach niniejszej pracy inżynierskiej może zapoczątkować nowe standardy na poziomie krajowym, europejskim i światowym, w zakresie niezbędnych działań minimalizujących wpływ siłowni na nietoperze – zamiast dotychczas stosowanych, z góry określonych wyłączeń na długi okres, mogłyby być wykorzystywane wyłączenia sterowane na bieżąco przez system – turbiny zatrzymywane byłyby tylko w czasie, kiedy większe liczebności nietoperzy faktycznie się pojawiają.

\section{Aktualne rozwiązania konkurencyjne}

Intensywny rozwój sieci neuronowych oraz internetu rzeczy (ang: IoT) przyczynił się do powstania podobnych rozwiązań. Na rynku światowym istnieją zbliżone pod kątem funkcjonalnym systemy, takie jak: DTBat i Fleximouse.

\section{IoT}

\section{Sieci neuronowe}

\section{Kooperacja z Bioseco S.A.}

\chapter{Metody pracy}

\section{Proces wytwórczy}

\subsection{Przyjęte podejście}

\subsection{Organizacja zespołu}

\subsection{Charakterystyka przyrostów}

\section{Srodowisko technologiczne}

\subsection{Technologie}

\subsection{Infrastuktura techniczna}

\subsection{Infrastuktura komunikacyjna i dokumentacja}

\section{Analiza zagrożeń}

\chapter{Prace badawczo-rozwojowe i projektowanie}

\section{Konsultacje z Bioseco S.A.}

\section{Konsultacje chiropterologiczne}

\section{Mikrofon ultradźwięków spełniający wymogi końcowego produktu}

\section{Odtwarzanie ultradźwięków}

\section{Głosy nietoperzy typu full-spectrum}

\subsection{Wytworzenie sztucznego głosu nietoperza w programie Audacity}

W celu przetestowania działania sprzętu rejestrującego i oprogramowania przetwarzającego zarejestrowane dźwięki, przeprowadzono doświadczenie polegające na sztucznym wytworzeniu dźwięków, które imitowały by głos nietoperza.
Pierwszym etapem było wygenerowanie w programie Audacity sinusoidalnej fali dźwiękowej z początkową częstotliwością 51 kHz i końcową częstotliwością 42 kHz, amplitudą początkową 0 i amplitudą końcową 1, interpolacją logarytmiczną oraz czasem trwania 25 ms (\ref{img:wykres_fali}).

\begin{figure}[h]
    \centering
    \includegraphics[width=0.8\textwidth]{sprz/wykres_fali}
    \caption{Wykres fali dźwiękowej}
    \label{img:wykres_fali}
\end{figure}

Uzyskany w ten sposób wykres fali dźwiękowej skopiowano dziesięciokrotnie i utworzono sekwencję 500 ms dźwięków, przed którą i po której wprowadzono 500 ms ciszy w celu łatwiejszego wyodrębnienia dźwięków po ich późniejszym zarejestrowaniu (\ref{img:wykres_fali_wielokrotnej}).

\begin{figure}[h]
    \centering
    \includegraphics[width=0.8\textwidth]{sprz/wykres_fali_wielokrotnej}
    \caption{Zwielokrotniony wykres fali}
    \label{img:wykres_fali_wielokrotnej}
\end{figure}

Następnie, do wyemitowania dźwięku doświadczenia niezbędne były urządzenia obsługujące co najmniej dwukrotnie wyższą częstotliwość niż emitowane dźwięki. W tym celu do komputera służącego jako generator dźwięku podłączono wzmacniacz iFi Zen Dac o maksymalnej częstotliwości odtwarzania 384 kHz oraz głośnik ultradźwiękowy Pettersson L400 (10-110 kHz). Do zarejestrowania wyemitowanego dźwięku posłużył mikrofon Pettersson M500-384 o częstotliwości próbkowania 384 kHz.

Program BatSound dołączony do mikrofonu posłużył do wizualizacji zarejestrowanego dźwięku.

\begin{figure}[h]
    \centering
    \includegraphics[width=0.8\textwidth]{sprz/batsound}
    \caption{Wykres zarejestrowanego dźwięku zwizualizowany w programie Batsound}
    \label{img:batsound}
\end{figure}

Powyższe doświadczenie dowiodło, iż mikrofon Pettersson M500-384 spełnia swoje zadanie. Mikrofon rejestruje dźwięki w zakresie niesłyszalnym dla człowieka i poprawnie wizualizuje zarejestrowane nagranie.

\subsection{Zebranie nagrań nietoperzy}

\chapter{Struktura produktu}

Struktura wytworzonego produktu została przedstawiona w podziale na poszczególne komponenty.

\section{Architektura systemu}

Na poniższych modelach przedstawiono najważniejsze elementy logiczne, urządzenia wejścia-wyjścia oraz obieg informacji w systemie (\ref{img:model_architektury}). W celu ułatwienia zrozumienia osadzenia systemu w rzeczywistości, architekturę zobrazowano również w postaci graficznej, łącznie z widokiem interfejsu (\ref{img:reprezentacja_graficzna}). 

\begin{figure}[h]
    \centering
    \includegraphics[width=0.8\textwidth]{sprz/model_architektury}
    \caption{Logiczny model architektury systemu}
    \label{img:model_architektury}
\end{figure}

\begin{figure}[h]
    \centering
    \includegraphics[width=0.8\textwidth]{sprz/reprezentacja_graficzna}
    \caption{Logiczny model architektury systemu}
    \label{img:reprezentacja_graficzna}
\end{figure}

\section{Hardware}

\section{Platforma pobierająca i przetwarzająca nagrania z mikrofonu ultradźwięków}

\section{Model sieci neuronowej}

\section{Baza danych}

\section{Aplikacja użytkownika}

\chapter{Realizacja sieci neuronowej}

\section{Przygotowanie danych}

\subsection{Czyszczenie}

\subsection{Dobór typu spektogramu}

\subsection{Cięcie}

\subsection{Augmentacja}

\section{Dobór sieci neuronowej}

\subsection{Przegląd literatury}

\subsection{Konsultacje}

\section{Dobór parametrów sieci neuronowej}

\subsection{Funkcje aktywacji}

\subsection{Funkcja błędu}

\subsection{Optymalizatory}

\subsection{Regularyzacja}

\subsection{Parametry warstw sieci}

\subsection{Transfer learning}

\section{Skuteczność sieci neuronowej}

\subsection{Macierz pomyłek}

\subsection{Metryki}

\subsection{Learning rate}

\chapter{Realizacja interfejsu użytkownika i bazy danych}

\section{Implementacja interfejsu użytkownika}

\section{Implementacja bazy danych}

\chapter{Testowanie}

\section{Testy terenowe produktu}

\section{Testy oprogramowania}

\subsection{Testy jednostkowe}

\subsection{Testy integracyjne}

\subsection{Testy funkcjonalne}

\subsection{Testy akceptacyjne}

\chapter{Przyszłość produktu i komercjalizacja}

\chapter{Podsumowanie}

\chapter{Wkład własny}

\section{Juliusz Orłowski}

\subsection{Wytworzenie sztucznego głosu nietoperza}

\subsection{Interfejs użytkownika}

\subsection{Baza danych}

\subsection{Książka projektu}

\section{Jakub Prucnal}

\subsection{Interfejs użytkownika}

\subsection{Baza danych}

\subsection{Przygotowanie danych do modelu}

\subsection{Model sieci neuronowej}

\subsection{Książka projektu}

\section{Magdalena Wybraniec}

\subsection{Zebranie nagrań nietoperzy}

\subsection{Interfejs użytkownika}

\subsection{Baza danych}

\subsection{Konsultacje chiropterologiczne}

\subsection{Przygotowanie danych do modelu}

\subsection{Model sieci neuronowej}

\subsection{Książka projektu}

\chapter{Załączniki}

\section{Dokument założeń wstępnych}

\begin{documenttable}[]
  \projectname{Batmonit}
  \customer{Bioseco S.A.}
  \contractor{PJATK}
  \projectteam{
    \begin{enumerate}
      \item Juliusz Orłowski
      \item Jakub Prucnal
      \item Magdalena Wybraniec
    \end{enumerate}
  }
  \projectlead{
    \begin{enumerate}
      \item Dawid Gradolewski
    \end{enumerate}
  }
  \documentname{Dokument Założeń Wstępnych}
  \documentowner{Jakub Prucnal}
  \projectsupervisor{dr Tadeusz Puźniakowski}
\end{documenttable}
\begin{center}
  \begin{tabular}{ |p{0.1\linewidth}|p{0.28\linewidth}|p{0.2\linewidth}|p{0.24\linewidth}|p{0.12\linewidth}| }
    \hline
    \multicolumn{5}{|c|}{\textbf{Historia dokumentu}} \\
    \hline
    \textbf{Wersja} & \textbf{Opis modyfikacji} & \textbf{Rozdział/strona} & \textbf{Autor modyfikacji} & \textbf{Data}\\
    \hline
    {1.0} & {Wstępna wersja} & {Całość} & {Magdalena Wybraniec \newline Jakub Prucnal} & {2021-11-21}\\
    \hline
    {2.0} & {Rozbudowanie opisu problemu, doprecyzowanie i drobne korekty pozostałych elementów} &
    {Całość} & {Magdalena Wybraniec} & {2021-12-14}\\
    \hline
  \end{tabular}
\end{center}

\begin{enumerate}[label=\textbf{\arabic*}.]
  \item \textbf{Opis problemu}
  
    Farmy wiatrowe mogą mieć znaczący wpływ na środowisko naturalne, w szczególności nietoperze. Wszystkie gatunki nietoperzy w Polsce są objęte ścisłą ochroną gatunkową i podlegają ochronie prawnej zgodnie z Rozporządzeniem Ministra Środowiska z dnia 16 grudnia 2016 r. w sprawie ochrony gatunkowej zwierząt. Tym samym inwestor realizujący inwestycję wiatrową przestrzegając prawa ochrony przyrody jest zobligowany uzyskać tzw. Decyzję Środowiskową (dalej: DŚ). Zawarte są w niej wszelkie informacje dotyczące:
    \begin{itemize}
      \item chiropterofauny danego obszaru zebrane w trakcie rocznego monitoringu przedrealizacyjnego,
      \item szacowanego wpływu inwestycji na nietoperze,
      \item działań zapobiegających i minimalizujących ewentualny wpływ inwestycji na nietoperze, tak by spełniała ona założenia dobrych praktyk i przepisy prawa – polskiego i międzynarodowego.
    \end{itemize}

    Dotychczas przyjętą praktyką w przypadku wykrycia zbyt dużych aktywności nietoperzy w monitoringu przedrealizacyjnym było wpisanie do DŚ obowiązkowych wyłączeń turbin w okresach, w których te zbyt duże aktywności wykryto. Jednak aktywność nietoperzy jest bardzo zmienna i po realizacji inwestycji niekoniecznie nietoperze będą w całych tych okresach aktywne na farmie, a przez to zagrożone. A wyłączenia turbin na całe długie okresy w roku wiążą się z ogromnymi stratami inwestorów. System opracowany w ramach niniejszej pracy inżynierskiej mógłby zapoczątkować nowe standardy na poziomie krajowym, europejskim i światowym, w zakresie niezbędnych działań minimalizujących wpływ wiatraków na nietoperze - zamiast dotychczas stosowanych z góry określonych wyłączeń na długi okres, można by stosować wyłączenia sterowane na bieżąco systemem – turbiny wyłączane by były tylko w czasie, kiedy większe liczby nietoperzy faktycznie się pojawiają.

    Problem ten dotyczy głównie farm lądowych, gdyż na farmach morskich aktywność nietoperzy jest zdecydowanie mniejsza.

    Zaprojektowany w ramach pracy inżynierskiej system ma za zadanie wytworzenie prototypu urządzenia zczytującego głos nietoperzy z mikrofonu ultradźwięków i ich analizę, tak by automatycznie wykrywać wystąpienie przelotu nietoperzy przy użyciu narzędzi sztucznej inteligencji, takich jak machine learning (dalej: ML) lub deep learning (dalej DL).

    Głównym udziałowcem i zarazem pomysłodawcą jest firma BIOSECO, której przedstawicielami są Dawid Gradolewski oraz Damian Dziak. Udziałowcami produktu, który może zostać wytworzony na dalszym etapie na bazie projektu inżynierskiego, są inwestorzy przygotowujący farmy wiatrowe w Polsce i na świecie, takie jak PGE, Energa, RWE, Polenergia, Iberdrola itd., a także twórcy monitoringów chiropterologicznych, Regionalne i Generalna Dyrekcja Ochrony Środowiska - posiadający tę opcję działań minimalizujących w swoim w arsenale.

    \begin{figure}[h]
        \centering
        \includegraphics[width=0.8\textwidth]{sprz/rich_picture}
        \label{img:rich_picture}
    \end{figure}

  \item \textbf{Cele systemu}
  
    System ma na celu zczytywać z mikrofonu ultradźwięki wydawane przez nietoperze, przesyłać je do uprzednio wytrenowanego modelu i sygnalizować, gdy nietoperze się pojawią, podając przy tym wykryty gatunek. Jednocześnie powinien zapisywać sygnały nietoperzy w bazie danych i wizualizować je w interfejsie użytkownika.  System po dopracowaniu i ewentualnej komercjalizacji przeznaczony będzie głównie dla firm posiadających lub zarządzających dużymi farmami wiatrowymi. Korzyści wynikające z zastosowania systemu to: ograniczenie śmiertelności nietoperzy na farmach wiatrowych i możliwość zrealizowania zapisów wynikających z pozwoleń prawnych na realizację konkretnego przedsięwzięcia.

  \item \textbf{Kontekst systemu}
  
    System będzie składał się z mikrofonu ultradźwięków wraz z urządzeniem rejestrującym dźwięk (komputerem), do którego będą trafiały wszystkie nagrania, tam analizowane i wystawiające API dla systemu wyłączającego turbinę wiatrową - zamontowane na turbinie wiatrowej.

  \item \textbf{Zakres systemu (funkcjonalność)}
  
    Do głównych funkcjonalności systemu będą należały:

    \begin{itemize}
      \item Rejestracja nagrań,
      \item Przesyłanie nagrań do urządzenia wewnętrznego, na którym są one przechowywane, analizowane i wstawiane do modelu ML/DL,
      \item Analiza nagrań z udziałem modelu ML/DL pod kątem występowania nietoperzy i ich gatunków,
      \item Wizualizacja nagrań nietoperzy w bazie danych poprzez interfejs, wraz z możliwością ich szczegółowego przeglądania.
    \end{itemize}

  \item \textbf{Wymagania jakościowe i inne}
  
    System powinien spełniać następujące wymagania:

    \begin{itemize}
        \item Przenośność urządzenia zewnętrznego,
        \item Niezawodne zasilanie urządzenia zewnętrznego,
        \item Adekwatna do warunków środowiskowych i technologicznych obudowa urządzenia zewnętrznego,
        \item Niezawodność połączenia pomiędzy zewnętrznym urządzeniem rejestrującym nagrania i urządzeniem przechowującym je,
        \item Możliwość wystawienia API i wykorzystania przez system wyłączający i włączający turbinę wiatrową.
    \end{itemize}

  \item \textbf{Wizja konstrukcyjna}
  
    Urządzenie zewnętrzne będzie składało się z mikrofonu Pettersson M500-384 (i opcjonalnie z obudową) oraz urządzenia rejestrującego nagranie (np. Raspberry Pi, komputer - do ustalenia), wyposażonego w zasilanie i pamięć. Kod pobierający nagranie z mikrofonu do urządzenia rejestrującego będzie utworzony pierwotnie w Pythonie. Kod modelu ML/DL będzie napisany w Pythonie z użyciem frameworka TensorFlow, PyTorch, lub innego.  

  \item \textbf{Ograniczenia}
  
    \begin{itemize}
      \item Czas trwania projektu jest ograniczony do momentu przekazania książki dyplomowej do dziekanatu uczelni PJATK.
      \item Ograniczona dostępność nagrań głosów nietoperzy w full-spectrum – możliwa sytuacja, gdy tworzenie projektu będzie odbywało się na podstawie nagrań przetworzonych.
      \item Aktywność nietoperzy ma miejsce od końca marca do października – w związku z tym brak możliwości nagrywania ich aktywności w trakcie semestru zimowego – a tym samym brak możliwości wykonania prób hardware’u przed kwieniem 2022.
    \end{itemize}

  \item \textbf{Słownik pojęć}
  
    Nagranie full-spectrum – nagranie rejestracji ultradźwięków w pełnym wymiarze

    ML (machine learning, uczenie maszynowe) – obszar sztucznej inteligencji wykorzystujący algorytmy i modele matematyczne, które automatycznie poprawiają się poprzez ekspozycję na dane, w celu prognozowania lub podejmowania decyzji dot. nowo dostarczanych danych

    DL (deep learning) – podzbiór ML wykorzystujący tzw. sieci neuronowe, w której pojedyncze neurony tworzą warstwy i stanowią de facto mnożenie wartości wejściowych z początkowo losowo przypisanymi wagami, gdzie wyniki z poszczególnych wejść są sumowane i przekazywane do tzw. funkcji aktywacji, które decydują czy i jakie wyniki przekazać do kolejnej warstwy neuronów.

\end{enumerate}

\section{Specyfikacja wymagań systemowych}

\begin{documenttable}[]
  \projectname{Batmonit}
  \customer{Bioseco S.A.}
  \contractor{PJATK}
  \projectteam{
    \begin{enumerate}
      \item Juliusz Orłowski
      \item Jakub Prucnal
      \item Magdalena Wybraniec
    \end{enumerate}
  }
  \projectlead{
    \begin{enumerate}
      \item Dawid Gradolewski
    \end{enumerate}
  }
  \documentname{Specyfikacja Wymagań Systemowych}
  \documentowner{Jakub Prucnal}
  \projectsupervisor{dr Tadeusz Puźniakowski}
\end{documenttable}
\begin{center}
  \begin{tabular}{ |p{0.1\linewidth}|p{0.28\linewidth}|p{0.2\linewidth}|p{0.24\linewidth}|p{0.12\linewidth}| }
    \hline
    \multicolumn{5}{|c|}{\textbf{Historia dokumentu}} \\
    \hline
    \textbf{Wersja} & \textbf{Opis modyfikacji} & \textbf{Rozdział/strona} & \textbf{Autor modyfikacji} & \textbf{Data}\\
    \hline
    {1.0} & {Wstępna wersja} & {Całość} & {Jakub Prucnal} & {2021-12-05}\\
    \hline
    {2.0} & {Całość niektórych punktów, poprawki reszty} & {punkty 2, 2.1, 2.3} & {Magdalena Wybraniec} & {2021-12-19}\\
    \hline
  \end{tabular}
\end{center}

\begin{enumerate}[label=\textbf{\arabic*}.]
  \item \textbf{Wprowadzenie – o dokumencie}
    \begin{enumerate}[label=\textbf{\arabic*}.]
      \item \textbf{Cel dokumentu}
      
        Zdefiniowanie wymagań na podstawie analizy otoczenia projektu oraz analizę potrzeb klienta.

      \item \textbf{Zakres dokumentu}
      
        Określenie udziałowców, zdefiniowanie wymagań, analiza otoczenia projektu.

      \item \textbf{Dokumenty powiązane}
      
        Karta Projektu (KP)
        Dokument Założeń Wstępnych (DZW)
        Rich Picture

      \item \textbf{Odbiorcy}
      
        \begin{itemize}
          \item Opiekun projektu Tadeusz Puźniakowski,
          \item Zespół projektowy,
          \item Przedstawiciele zleceniodawcy: Dawid Gradolewski oraz Damian Dziak
          \item Zleceniobiorca
        \end{itemize}  

      \item \textbf{Słownik pojęć}
      
        Nagranie full-spectrum – nagranie rejestracji ultradźwięków w pełnym wymiarze
        ML (machine learning, uczenie maszynowe) – obszar sztucznej inteligencji wykorzystujący algorytmy i modele matematyczne, które automatycznie poprawiają się poprzez ekspozycję na dane, w celu prognozowania lub podejmowania decyzji dot. nowo dostarczanych danych 
        DL (deep learning) – podzbiór ML wykorzystujący tzw. sieci neuronowe, w której pojedyncze neurony tworzą warstwy i stanowią de facto mnożenie wartości wejściowych z początkowo losowo przypisanymi wagami, gdzie wyniki z poszczególnych wejść są sumowane i przekazywane do tzw. funkcji aktywacji, które decydują czy i jakie wyniki przekazać do kolejnej warstwy neuronów.      

    \end{enumerate}
  \item \textbf{Projekt w kontekście}
  \item \textbf{Wymagania}
  \item \textbf{Odwołania do literatury}
\end{enumerate}

\section{Diagram przypadków użycia}

\chapter{Karty udziałowca}


\begin{stakeholder}[label={tab:stakeholder:someholder},caption={Przykładowy opis udzialowca}]
    \id{jednoznaczny symbol np. UOB 01, UOB 02 ... dla udziałowców ożywionych bezpośrednich, UNP 01... dla nieożywionych pośrednich itd.}
    \name{nazwa udziałowca}
    \descr{opis udziałowca}
    \type{ożywiony/nieożywiony, bezpośredni/pośredni}
    \viewpoint{z jakiej perspektywy patrzy udziałowiec np. technicznej, ekonomicznej, operatora systemu itp.}
    \limitations{ograniczenia udziałowca np. administrator nie powinien specyfikować wymagań finansowych}
    \requ{tu tylko symbole wymagań wyspecyfikowanych w rozdziale 3}
\end{stakeholder}

\chapter{Wymagania wszelakie}

Na tabeli \ref{tab:requirements:general} pokazano jak można definiować wymagania ogólne lub dziedzinowe.

\begin{requirementstab}[label={tab:requirements:general},caption={Przykładowe wymaganie ogólne lub dziedzinowe}]
    \id{jednoznaczny symbol np. WO1, WO2 .. }
    \priority{ważność wymagania, np. wg skali MoSCoW: M – must (musi być) S – should (powinno być) C – could (może być) W – won't (nie będzie – nie będzie implementowane w danym wydaniu, ale może być rozpatrzone w przyszłości )}
    \name{krótki opis}
    \descr{opis szczegółowy, należy dążyć do tego, żeby wszystkie znane na ten moment szczegóły wymagania zostały wydobyte i wyspecyfikowane}
    \sholder{nazwa udziałowca, który podał wymaganie}
    \reqrelated{wymagania zależne i uszczegóławiające – odesłanie poprzez identyfikator}
\end{requirementstab}

Teraz czas na wymagania funkcjonalne, na przykład \ref{tab:requirements:func1}

\begin{requirementstab}[label={tab:requirements:func1},caption={Pryzkładowa tabela z wymaganiami na interfejs z otoczeniem}]
    \id{jednoznaczny symbol np. FO1, FO2 .. }
    \priority{ważność wymagania, np. wg skali MoSCoW: M – must (musi być) S – should (powinno być) C – could (może być) W – won't (nie będzie – nie będzie implementowane w danym wydaniu, ale może być rozpatrzone w przyszłości )}
    \name{krótki opis}
    \descr{opis szczegółowy, należy dążyć do tego, żeby wszystkie znane na ten moment szczegóły wymagania zostały wydobyte i wyspecyfikowane

    Można zastosować opis jak w User Story
    \begin{itemize}
        \item Jako.. (konkretny użytkownik systemu)
        \item chcę... (pożądana cecha lub problem, który trzeba rozwiązać)
        \item bo wtedy/ponieważ… (korzyść płynąca z ukończenia story)
    \end{itemize}
    }
    \acceptcrit{Warunki Satysfakcji (Szczegóły dodane na potrzeby  testów akceptacyjnych)}
    \inputdata{uzupełniane w trakcie sprintu – dane wejściowe, związane z wymaganiem}
    \preconditions{ uzupełniane w trakcie sprintu – warunki, które muszą być prawdziwe przed wywołaniem operacji}
    \postconditions{ uzupełniane w trakcie sprintu – warunki, które muszą być prawdziwe po wywołaniu operacji}
    \exceptions{ uzupełniane w trakcie sprintu – niepożądane sytuacje i sposoby ich obsługi}
    \implementation{ uzupełniane w trakcie sprintu – opis sposobu realizacji}
    \sholder{nazwa udziałowca, który podał wymaganie}
    \reqrelated{wymagania zależne i uszczegóławiające – odesłanie poprzez identyfikator}
\end{requirementstab}


Natomiast tabela \ref{tab:requirements:env1} pokazuje wymagania na interfejs z otoczeniem.

\begin{requirementstab}[label={tab:requirements:env1},caption={Pryzkładowa tabela z wymaganiami na interfejs z otoczeniem}]
    \id{jednoznaczny symbol np. IO1, IO2 .. }
    \priority{ważność wymagania, np. wg skali MoSCoW: M – must (musi być) S – should (powinno być) C – could (może być) W – won't (nie będzie – nie będzie implementowane w danym wydaniu, ale może być rozpatrzone w przyszłości )}
    \name{krótki opis}
    \descr{opis szczegółowy, należy dążyć do tego, żeby wszystkie znane na ten moment szczegóły wymagania zostały wydobyte i wyspecyfikowane}
    \acceptcrit{Warunki Satysfakcji (Szczegóły dodane na potrzeby  testów akceptacyjnych)}
    \inputdata{uzupełniane w trakcie sprintu – dane wejściowe, związane z wymaganiem}
    \preconditions{ uzupełniane w trakcie sprintu – warunki, które muszą być prawdziwe przed wywołaniem operacji}
    \postconditions{ uzupełniane w trakcie sprintu – warunki, które muszą być prawdziwe po wywołaniu operacji}
    \exceptions{ uzupełniane w trakcie sprintu – niepożądane sytuacje i sposoby ich obsługi}
    \implementation{ uzupełniane w trakcie sprintu – opis sposobu realizacji}
    \sholder{nazwa udziałowca, który podał wymaganie}
    \reqrelated{wymagania zależne i uszczegóławiające – odesłanie poprzez identyfikator}
\end{requirementstab}

Tabela dotycząca wymagań pozafunkcjonalnych \ref{tab:requirements:nonfunc1} jest także widoczne.

\begin{requirementstab}[label={tab:requirements:nonfunc1},caption={Pryzkładowa tabela z wymaganiami pozafunkcjonalnymi}]
    \id{jednoznaczny symbol np. NFO1, NFO2 .. }
    \priority{ważność wymagania, np. wg skali MoSCoW: M – must (musi być) S – should (powinno być) C – could (może być) W – won't (nie będzie – nie będzie implementowane w danym wydaniu, ale może być rozpatrzone w przyszłości )}
    \name{krótki opis}
    \descr{opis szczegółowy, należy dążyć do tego, żeby wszystkie znane na ten moment szczegóły wymagania zostały wydobyte i wyspecyfikowane}
    \acceptcrit{Warunki Satysfakcji (Szczegóły dodane na potrzeby  testów akceptacyjnych)}
    \sholder{nazwa udziałowca, który podał wymaganie}
    \reqrelated{wymagania zależne i uszczegóławiające – odesłanie poprzez identyfikator}
\end{requirementstab}


\printbibliography[title={Bibliografia}, heading=bibintoc]




\end{document}
